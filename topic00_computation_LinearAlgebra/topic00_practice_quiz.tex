\documentclass[10pt]{exam}

\usepackage[margin=1in]{geometry}
\usepackage{amsmath}
\usepackage{amssymb}
\usepackage{amsthm}
\usepackage{mathtools}
\usepackage{bm}

\usepackage{color}
\usepackage{colortbl}
\definecolor{deepblue}{rgb}{0,0,0.5}
\definecolor{deepred}{rgb}{0.6,0,0}
\definecolor{deepgreen}{rgb}{0,0.5,0}
\definecolor{gray}{rgb}{0.7,0.7,0.7}

\usepackage{hyperref}
\hypersetup{
  colorlinks   = true, %Colours links instead of ugly boxes
  urlcolor     = black, %Colour for external hyperlinks
  linkcolor    = blue, %Colour of internal links
  citecolor    = blue  %Colour of citations
}

%%%%%%%%%%%%%%%%%%%%%%%%%%%%%%%%%%%%%%%%%%%%%%%%%%%%%%%%%%%%%%%%%%%%%%%%%%%%%%%%

\newcommand*{\hl}[1]{\colorbox{yellow}{#1}}

\newcommand*{\answerLong}[2]{
    \ifprintanswers{\hl{#1}}
\else{#2}
\fi
}

\newcommand*{\answer}[1]{\answerLong{#1}{~}}

\newcommand*{\TrueFalse}[1]{%
\ifprintanswers
    \ifthenelse{\equal{#1}{T}}{%
        %\hl{\textbf{TRUE}}\hspace*{14pt}False
        \hl{\texttt{True}}\hspace*{20pt}\texttt{False}\hspace*{20pt}\texttt{Open}
    }{
        \ifthenelse{\equal{#1}{F}}{
        %True\hspace*{14pt}\hl{\textbf{FALSE}}
        \texttt{True}\hspace*{20pt}\hl{\texttt{False}}\hspace*{20pt}\texttt{Open}
        }
        {
            \texttt{True}\hspace*{20pt}{\texttt{False}}\hspace*{20pt}\hl{\texttt{Open}}
        }
    }
\else
    \texttt{True}\hspace*{20pt}\texttt{False}\hspace*{20pt}\texttt{Open}
\fi
} 
%% The following code is based on an answer by Gonzalo Medina
%% https://tex.stackexchange.com/a/13106/39194
\newlength\TFlengthA
\newlength\TFlengthB
\settowidth\TFlengthA{\hspace*{1.8in}}
\newcommand\TFQuestion[2]{%
    \setlength\TFlengthB{\linewidth}
    \addtolength\TFlengthB{-\TFlengthA}
    \noindent
    \parbox[t]{\TFlengthA}{\TrueFalse{#1}}\parbox[t]{\TFlengthB}{#2}
    \vspace{0.25in}
}

%%%%%%%%%%%%%%%%%%%%%%%%%%%%%%%%%%%%%%%%%%%%%%%%%%%%%%%%%%%%%%%%%%%%%%%%%%%%%%%%

\theoremstyle{definition}
\newtheorem{problem}{Problem}
\newtheorem{note}{Note}
\newtheorem{theorem}{Theorem}
\newtheorem{defn}{Definition}
\newtheorem{refr}{References}
\newcommand{\E}{\mathbb E}
\newcommand{\R}{\mathbb R}
\DeclareMathOperator{\nnz}{nnz}
\DeclareMathOperator{\determinant}{det}
\DeclareMathOperator{\Var}{Var}
\DeclareMathOperator{\rank}{rank}
\DeclareMathOperator*{\argmin}{arg\,min}
\DeclareMathOperator*{\argmax}{arg\,max}

\newcommand{\I}{\mathbf I}
\newcommand{\Q}{\mathbf Q}
\newcommand{\p}{\mathbf P}
\newcommand{\pb}{\bar {\p}}
\newcommand{\pbb}{\bar {\pb}}
\newcommand{\pr}{\bm \pi}

\newcommand{\trans}[1]{{#1}^{T}}
\newcommand{\loss}{\ell}
\newcommand{\w}{\mathbf w}
\newcommand{\x}{\mathbf x}
\newcommand{\y}{\mathbf y}
\newcommand{\lone}[1]{{\lVert {#1} \rVert}_1}
\newcommand{\ltwo}[1]{{\lVert {#1} \rVert}_2}
\newcommand{\lp}[1]{{\lVert {#1} \rVert}_p}
\newcommand{\linf}[1]{{\lVert {#1} \rVert}_\infty}
\newcommand{\lF}[1]{{\lVert {#1} \rVert}_F}

\newcommand{\ignore}[1]{}

%%%%%%%%%%%%%%%%%%%%%%%%%%%%%%%%%%%%%%%%%%%%%%%%%%%%%%%%%%%%%%%%%%%%%%%%%%%%%%%%

\printanswers
\begin{document}


\begin{center}
{
\Huge
    Practice Quiz: Computational Linear Algebra
}
\end{center}

\begin{note}
Your real quiz will have 4 problems following the format below.
You will be allowed to use any hand written or electronic reference material that you would like,
including websites like Wolfram Alpha and ChatGPT.
The only restriction is that you will not be able to communicate with other students.

HINT:
Ensure that you pay careful attention to the formal definitions of asymptotic notation in your responses.
\end{note}

\begin{problem}
    For each statement below,
    circle \texttt{True} if the statement is known to be true,
    \texttt{False} if the statement is known to be false,
    and \texttt{Open} if the statement reduces to an open problem.
    You will receive +1 point for each correct answer,
    -1 point for each incorrect anwser,
    and 0 points for each blank answer.


\begin{enumerate}
    \item\TFQuestion{T}{Let $f(n) = 1/(1+n)$. Then $f = \Omega(n^{-2})$.}
    \item\TFQuestion{T}{Let $f(n) = n^3 + 1/n$. Then $f = \Omega(n^{3})$.}
    \item\TFQuestion{T}{Let $f(n) = 1/n$. Then $f = \Omega(n^{-2})$.}
    \item\TFQuestion{F}{Let $f(n) = 2^n$. Then $f = \Theta(3^n)$.}
    \item\TFQuestion{T}{Let $f(a,b) = 5a^2 + 3ab$. Then $f = O(a^2b)$.}
    \item\TFQuestion{T}{Let $f(a,b) = 5a^2 + 3ab$. Then $f = O(a^2 + ab)$.}
\item\TFQuestion{T}{Let $A$ and $B$ be $n\times n$ matrices. The fastest algorithm for computing the matrix product $AB$ has runtime $O(n^4)$.}
\item\TFQuestion{O}{Let $A$ and $B$ be $n\times n$ matrices. The fastest algorithm for computing the matrix product $AB$ has runtime $\Theta(n^2 \log n)$.}
\item\TFQuestion{O}{Let $A$ and $B$ be $n\times n$ matrices. The fastest algorithm for computing the matrix product $AB$ has runtime $\Omega(n^{2.1})$.}
\item\TFQuestion{T}{Let $A$ and $B$ be $n\times n$ matrices. The fastest algorithm for computing the matrix product $AB$ has runtime $\Omega(n^{2})$.}
\item\TFQuestion{F}{Let $A$ and $B$ be $n\times n$ matrices. The fastest algorithm for computing the matrix product $AB$ has runtime $\Omega(n^{2.5})$.}
\item\TFQuestion{O}{The matrix chain ordering problem can be solved in time $\Theta(n)$.}
\item\TFQuestion{T}{Computing $\trans\x\x$ is faster than computing $\x\trans\x$.}
\item\TFQuestion{T}{There exists a family of $n \times n$ matrices $A$ and $B$ with $\nnz(A) = O(1)$ and $\nnz(B) = O(1)$ such that the product $AB$ satisfies $\nnz(AB) = O(1)$.}
\item\TFQuestion{F}{There exists a family of $n \times n$ matrices $A$ and $B$ with $\nnz(A) = O(1)$ and $\nnz(B) = O(1)$ such that the product $AB$ satisfies $\nnz(AB) = \Omega(n)$.}
\item\TFQuestion{T}{Let $A$ and $B$ be arbitrary $n\times n$ matrices satisfying $\nnz(A) = O(1)$ and $\nnz(B) = O(1)$. Then the product $AB$ must satisfy $\nnz(AB) = O(1)$.}
\item\TFQuestion{T}{Let $A$ and $B$ be arbitrary $n\times n$ matrices satisfying $\nnz(A) = \Omega(n)$ and $\nnz(B) = \Omega(1)$. Then the product $AB$ must satisfy $\nnz(AB) = O(n^2)$.}
\item\TFQuestion{F}{Let $A$ be an $n\times n$ matrix.  Then $\nnz(A) = \Omega(n^3)$.}
\item\TFQuestion{T}{Let $A$ be an $n\times n$ matrix.  Then $\nnz(A) = O(n^3)$.}
\item\TFQuestion{T}{Let $A:n\times n$ and $\x:n$, then the best possible runtime of computing $(A\trans A)^{-1}\x$ is $\Omega(n^2)$.}
\item\TFQuestion{T}{Let $\x:n$, then the best possible runtime of computing $\x\trans \x \x\trans \x$ is $O(n^2)$.}
\item\TFQuestion{F}{Let $\x:n$, then the best possible runtime of computing $\x\trans \x \x\trans \x$ is $\Theta(n)$.}
\item\TFQuestion{F}{Let $A:a\times b$ and $\x:b$, then the best possible runtime of computing $\lF{A\x}^2$ is $\Omega(n^3)$.}
\end{enumerate}
\end{problem}

\end{document}




