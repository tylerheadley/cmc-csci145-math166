\documentclass[10pt]{exam}

\usepackage[margin=1in]{geometry}
\usepackage{amsmath}
\usepackage{amssymb}
\usepackage{amsthm}
\usepackage{mathtools}
\usepackage{bm}
\usepackage[normalem]{ulem}
\usepackage{stmaryrd}

\usepackage{color}
\usepackage{colortbl}
\definecolor{deepblue}{rgb}{0,0,0.5}
\definecolor{deepred}{rgb}{0.6,0,0}
\definecolor{deepgreen}{rgb}{0,0.5,0}
\definecolor{gray}{rgb}{0.7,0.7,0.7}

\usepackage{hyperref}
\hypersetup{
  colorlinks   = true, %Colours links instead of ugly boxes
  urlcolor     = black, %Colour for external hyperlinks
  linkcolor    = blue, %Colour of internal links
  citecolor    = blue  %Colour of citations
}

%%%%%%%%%%%%%%%%%%%%%%%%%%%%%%%%%%%%%%%%%%%%%%%%%%%%%%%%%%%%%%%%%%%%%%%%%%%%%%%%

\newcommand*{\hl}[1]{\colorbox{yellow}{#1}}

\newcommand*{\answerLong}[2]{
    \ifprintanswers{\hl{#1}}
\else{#2}
\fi
}

\newcommand*{\answer}[1]{\answerLong{#1}{~}}

\newcommand*{\TrueFalse}[1]{%
\ifprintanswers
    \ifthenelse{\equal{#1}{T}}{%
        %\hl{\textbf{TRUE}}\hspace*{14pt}False
        \hl{\texttt{True}}\hspace*{20pt}\texttt{False}\hspace*{20pt}\texttt{Open}
    }{
        \ifthenelse{\equal{#1}{F}}{
        %True\hspace*{14pt}\hl{\textbf{FALSE}}
        \texttt{True}\hspace*{20pt}\hl{\texttt{False}}\hspace*{20pt}\texttt{Open}
        }
        {
            \texttt{True}\hspace*{20pt}{\texttt{False}}\hspace*{20pt}\hl{\texttt{Open}}
        }
    }
\else
    \texttt{True}\hspace*{20pt}\texttt{False}\hspace*{20pt}\texttt{Open}
\fi
} 
%% The following code is based on an answer by Gonzalo Medina
%% https://tex.stackexchange.com/a/13106/39194
\newlength\TFlengthA
\newlength\TFlengthB
\settowidth\TFlengthA{\hspace*{1.8in}}
\newcommand\TFQuestion[2]{%
    \setlength\TFlengthB{\linewidth}
    \addtolength\TFlengthB{-\TFlengthA}
    \noindent
    \parbox[t]{\TFlengthA}{\TrueFalse{#1}}\parbox[t]{\TFlengthB}{#2}
    \vspace{0.25in}
}

%%%%%%%%%%%%%%%%%%%%%%%%%%%%%%%%%%%%%%%%%%%%%%%%%%%%%%%%%%%%%%%%%%%%%%%%%%%%%%%%

\theoremstyle{definition}
\newtheorem{problem}{Problem}
\newtheorem{theorem}{Theorem}
\newtheorem{defn}{Definition}
\newtheorem{refr}{References}
\newcommand{\E}{\mathbb E}
\newcommand{\R}{\mathbb R}
\DeclareMathOperator{\nnz}{nnz}
\DeclareMathOperator{\sign}{sign}
\DeclareMathOperator{\determinant}{det}
\DeclareMathOperator{\Var}{Var}
\DeclareMathOperator{\rank}{rank}
\DeclareMathOperator*{\argmin}{arg\,min}
\DeclareMathOperator*{\argmax}{arg\,max}

\newcommand{\I}{\mathbf I}
\newcommand{\Q}{\mathbf Q}
\newcommand{\p}{\mathbf P}
\newcommand{\pb}{\bar {\p}}
\newcommand{\pbb}{\bar {\pb}}
\newcommand{\pr}{\bm \pi}

\newcommand{\trans}[1]{{#1}^{T}}
\newcommand{\loss}{\ell}
\newcommand{\w}{\mathbf w}
\newcommand{\x}{\mathbf x}
\newcommand{\y}{\mathbf y}
\newcommand{\lone}[1]{{\lVert {#1} \rVert}_1}
\newcommand{\ltwo}[1]{{\lVert {#1} \rVert}_2}
\newcommand{\lp}[1]{{\lVert {#1} \rVert}_p}
\newcommand{\linf}[1]{{\lVert {#1} \rVert}_\infty}
\newcommand{\lF}[1]{{\lVert {#1} \rVert}_F}

\newcommand{\Ein}{E_{\text{in}}}
\newcommand{\Eout}{E_{\text{out}}}
\newcommand{\Etest}{E_{\text{test}}}
%\newcommand{\I}{\mathbf I}
%\newcommand{\Q}{\mathbf Q}
%\newcommand{\p}{\mathbf P}
%\newcommand{\pb}{\bar {\p}}
%\newcommand{\pbb}{\bar {\pb}}
%\newcommand{\pr}{\bm \pi}
\newcommand{\mH}{m_{\mathcal H}}
\newcommand{\dvc}{{d_{\text{VC}}}}
\newcommand{\HH}[1]{\mathcal H_{\text{#1}}}
\newcommand{\Hbinary}{\HH_{\text{binary}}}
\newcommand{\Haxis}{\HH_{\text{axis}}}
\newcommand{\Hperceptron}{\HH_{\text{perceptron}}}


\newcommand{\ignore}[1]{}

%%%%%%%%%%%%%%%%%%%%%%%%%%%%%%%%%%%%%%%%%%%%%%%%%%%%%%%%%%%%%%%%%%%%%%%%%%%%%%%%

%\printanswers

\begin{document}


\begin{center}
    {
\Large
    Quiz: Chapter 3 (Section 3.4)

    \vspace{0.1in}
}
\end{center}

%\vspace{0.15in}
%\noindent
%\textbf{Total Score:} ~~~~~~~~~~~~~~~/$2^2$

\vspace{0.2in}
\noindent
\textbf{Printed Name:}

\noindent
\rule{\textwidth}{0.1pt}
\vspace{0.15in}

\noindent
\textbf{Quiz rules:}
\begin{enumerate}
    \item You MAY use:
        \begin{enumerate}
            \item any notes (handwritten, printed, or electronic),
            \item any computer programs (including websites like WolframAlpha or ChatGPT), and
            \item any additional scratch paper.
        \end{enumerate}
    \item You MAY NOT communicate with another student.
    \item If you finish the quiz early, stay seated.
        I will collect all the quizzes at the same time.
\end{enumerate}

\begin{problem}
    For each statement below,
    circle \texttt{True} if the statement is known to be true,
    \texttt{False} if the statement is known to be false,
    and \texttt{Open} if the statement is not known to be either true or false.
    You will receive +2 point for each correct answer,
    \textbf{-1 points for each incorrect answer},
    and 0 points for each blank answer.

\begin{enumerate}
    \item\TFQuestion{T}{Let $\HH{perceptron}$ be the perceptron hypothesis class and let $\HH{stump}$ be the decision stump hypothesis class.  If $k$ is a breakpoint for $\HH{stump}$, then $k$ is guaranteed to be a breakpoint for $\HH{perceptron}$ as well.}

    \item\TFQuestion{F}{Let $\HH{$\Phi_2$}$ be the perceptron hypothesis class with the 2nd degree polynomial feature embedding, and $\HH{$\Phi_4$}$ be perceptron hypothesis class with the 4th degree polynomial feature embedding.
        Let $g_{\Phi_2}\in\HH{$\Phi_2$}$ and $g_{\Phi_4} \in \HH{$\Phi_4$}$ be the empirical risk minimizers.
       It is guaranteed that $\Ein(g_{\Phi_2}) \le \Ein(g_{\Phi_4})$.}

    \item\TFQuestion{F}{Let $\HH{$\Phi_2$}$ be the perceptron hypothesis class with the PCA feature embedding with output dimension 2, and $\HH{$\Phi_4$}$ be perceptron hypothesis class with the PCA feature embedding with output dimension 4.
        Let $g_{\Phi_2}\in\HH{$\Phi_2$}$ and $g_{\Phi_4} \in \HH{$\Phi_4$}$ be the empirical risk minimizers.
       VC theory predicts that with high probability, the generalization error of $g_{\Phi_2}$ will be less than the generalization error of $g_{\Phi_4}$.}

   \item\TFQuestion{F}{VC theory predicts that when using the $\HH{axis2}$ hypothesis class, centering your data points results in a better generalization error with high probability.}

\end{enumerate}
\end{problem}
\end{document}




